\documentclass[11pt,a4paper]{letter}
\usepackage[utf8]{inputenc}
\usepackage[T1]{fontenc}
\usepackage[ngerman]{babel}
\usepackage{geometry}
\usepackage{hyperref}

% Page margins
\geometry{
  left=2.5cm,
  right=2.5cm,
  top=2cm,
  bottom=2cm
}

% Sender information
\address{Erfan Taghvaei \\
Kaiserslautern, Deutschland \\
+49 157 35338285 \\
etaghvaei0098@gmail.com}

% Date
\date{\today}

\begin{document}

% Recipient
\begin{letter}{%
adesso SE \\
Recruiting Team \\
Deutschland
}

\opening{Sehr geehrtes Recruiting-Team,}

mit großer Begeisterung bewerbe ich mich auf die Position als Full Stack Engineer bei adesso. Die Möglichkeit, in IT-Projekten von Anfang bis Ende mitzuwirken, innovative Lösungen zu entwickeln und gemeinsam mit einem dynamischen Team Software zu gestalten, die Kunden begeistert, spricht mich sehr an.

\textbf{Meine Erfahrung und Qualifikation:}

Ich bringe zwei Jahre Berufserfahrung als React Native Entwickler bei Neocosmo in Deutschland mit, wo ich an verschiedenen Kundenprojekten für Unternehmen wie Festo, KVNO, Riniger und die Hochschule Kempten gearbeitet habe. In dieser Zeit habe ich gelernt, IT-Projekte von der Anforderungsanalyse bis zur erfolgreichen Umsetzung zu begleiten. Besonders prägend war ein Projekt, bei dem ich über zwei Monate eigenverantwortlich ein kritisches Ticket bearbeitet und in direkter Zusammenarbeit mit dem Kunden alle Anforderungen erfolgreich umgesetzt habe.

\textbf{Technische Kompetenzen als Full Stack Engineer:}

\begin{itemize}
  \item \textbf{Frontend-Entwicklung}: Zwei Jahre Berufserfahrung mit React Native und moderne Webanwendungen mit React. Entwicklung responsiver Benutzeroberflächen und Single-Page Applications
  \item \textbf{JavaScript}: Umfangreiche praktische Erfahrung in der JavaScript-Entwicklung, fundierte Kenntnisse in modernen ES6+ Features
  \item \textbf{Backend-Entwicklung}: Fullstack-Projekte mit Flask (Python), REST-API-Entwicklung, Server-Client-Architektur
  \item \textbf{Webtechnologien}: Erfahrung mit modernen Frontend-Frameworks, Build-Tools und PWA-Entwicklung
  \item \textbf{Software-Architektur}: Konzeption und Realisierung von skalierbaren Lösungen in verschiedenen Projekten
  \item \textbf{Qualitätssicherung}: Code Reviews, Unit Testing mit modernen Testing-Frameworks (Detox, Storybook), Continuous Integration
  \item \textbf{Agile Entwicklung}: Zwei Jahre Erfahrung in agilen Teams mit Sprint-Planning, Daily Standups und iterativer Entwicklung
  \item \textbf{Git \& DevOps}: Tägliche Arbeit mit Git, GitLab CI/CD-Pipelines, Docker-Containerisierung, NGINX-Konfiguration
\end{itemize}

\textbf{Projekte und praktische Erfahrung:}

Zusätzlich zu meiner Berufserfahrung habe ich mehrere Fullstack-Projekte eigenständig entwickelt:

\begin{itemize}
  \item \textbf{RecallCards}: Vollständige Flashcard-Applikation mit React-Frontend, Flask-Backend, Auth0-Authentication, Redis-Session-Management, Docker-Containerisierung und GitHub Actions CI/CD
  \item \textbf{AAS Sanity Projekt}: Webanwendung mit Bootstrap-Frontend und Flask-Backend, inklusive Multithreading-Optimierung und automatisierter Datenverarbeitung
\end{itemize}

Diese Projekte zeigen meine Fähigkeit, moderne Technologien effektiv einzusetzen und komplexe Anwendungen von Grund auf zu entwickeln.

\textbf{Akademischer Hintergrund:}

Aktuell studiere ich im Master of Computer Science an der RPTU Kaiserslautern mit Schwerpunkten in Software Engineering, Data Visualization und Network Security. Mein Bachelor in Software Engineering von der Tehran Azad University umfasst Kurse wie Compiler Design, Data Structures and Algorithms, Object-Oriented Programming und Operating Systems.

\textbf{Was mich auszeichnet:}

\begin{itemize}
  \item \textbf{Kundenorientierung}: Erfahrung in direkter Kundenkommunikation und Anforderungsanalyse
  \item \textbf{Lösungsorientierung}: Proaktive Problemlösung und analytisches Denken
  \item \textbf{Teamfähigkeit}: Enge Zusammenarbeit mit cross-funktionalen Teams und Stakeholdern
  \item \textbf{Lernbereitschaft}: Kontinuierliche Weiterbildung -- MongoDB Associate Developer Zertifizierung (2024)
  \item \textbf{Sprachkenntnisse}: Sehr gute Deutschkenntnisse (C1) und fließende Englischkenntnisse (C2) für sichere Kommunikation
  \item \textbf{Eigeninitiative}: Programmieren auch in der Freizeit, ständige Weiterentwicklung
\end{itemize}

\textbf{Warum adesso:}

Ihre Unternehmenskultur, die Wert auf gemeinsamen Erfolg und Teamgeist legt, spricht mich sehr an. Die über 400 Trainingsangebote und die digitale Lernplattform zeigen, dass adesso die Entwicklung seiner Mitarbeiter ernst nimmt -- genau das Umfeld, in dem ich wachsen möchte. Die Möglichkeit, an herausfordernden Projekten zu arbeiten und gleichzeitig von erfahrenen Kollegen zu lernen, ist für mich die ideale Kombination.

Ich bin überzeugt, dass ich mit meiner Fullstack-Erfahrung, meiner Leidenschaft für sauberen Code und meinem Engagement eine wertvolle Bereicherung für Ihr Team sein kann. Die Arbeit an innovativen Kundenprojekten bei adesso wäre für mich eine ideale Gelegenheit, meine Fähigkeiten einzubringen und weiterzuentwickeln.

Sehr gerne würde ich Sie in einem persönlichen Gespräch von meinen Fähigkeiten überzeugen und mehr über die spannenden Projekte bei adesso erfahren.

\closing{Mit freundlichen Grüßen,}

\end{letter}

\end{document}
