\documentclass[11pt,a4paper]{letter}
\usepackage[utf8]{inputenc}
\usepackage[T1]{fontenc}
\usepackage[ngerman]{babel}
\usepackage{geometry}
\usepackage{hyperref}

% Page margins
\geometry{
  left=2.5cm,
  right=2.5cm,
  top=2cm,
  bottom=2cm
}

\begin{document}

% Sender information
\address{Erfan Taghvaei \\
Kaiserslautern, Deutschland \\
+49 157 35338285 \\
etaghvaei0098@gmail.com}

% Date
\date{\today}

% Recipient
\begin{letter}{%
SAP SE \\
Walldorf \\
Deutschland
}

\opening{Sehr geehrtes SAP Product Experience Team,}

mit großem Interesse bewerbe ich mich auf die Position als Working Student (f/m/d) - Software Development Product Experience \& Analytics bei SAP. Die Möglichkeit, an der Entwicklung von Tools für datengetriebene Entscheidungsfindung mitzuwirken und dabei sowohl Frontend- als auch Backend-Entwicklung zu betreiben, motiviert mich außerordentlich.

Ich studiere aktuell im Master of Computer Science an der RPTU Kaiserslautern mit Schwerpunkt Software Engineering und Data Visualization. Meine akademische Ausbildung umfasst relevante Kurse wie Data Visualization, Foundations of Software Engineering sowie Social Web Mining (Note: 1.7). Besonders hervorheben möchte ich mein DFKI Smart Factory Projekt, bei dem ich eine vollständige Full-Stack-Anwendung mit modernen Webtechnologien und umfassender Dokumentation entwickelt habe.

Meine praktische Erfahrung als React Native Entwickler bei Neocosmo (Feb 2022 - Feb 2024) hat mir wertvolle Einblicke in die Software-Entwicklung gegeben. Ich habe eigenverantwortlich komplexe Tickets bearbeitet, dabei direkt mit Kunden kommuniziert und erfolgreich JavaScript-Code zu TypeScript konvertiert. Diese Erfahrung hat meine Fähigkeiten in der API-Entwicklung, Datenmodellierung und kontinuierlichen Kommunikation gestärkt.

Meine technischen Kompetenzen umfassen fundierte Programmierkenntnisse in JavaScript und TypeScript, die ich sowohl in meinem DFKI-Projekt als auch in meiner Forschungsarbeit "COVID-19 personal protective equipment detection using real-time deep learning methods" (arXiv:2103.14878) eingesetzt habe. In dieser Arbeit habe ich mit YOLO und SSD MobileNet-Algorithmen gearbeitet und eine Genauigkeit von 90.6\% erreicht. Diese Erfahrung mit Datenverarbeitung und Visualisierung ist besonders relevant für Ihre Telemetrie-Dashboards und datengetriebene Analysen.

Ich verfüge über umfassende Erfahrung mit Git und GitHub, die ich sowohl in meinem DFKI-Projekt als auch in meiner Berufserfahrung bei Neocosmo eingesetzt habe. In meinem Circular Economy Seminar habe ich über 30 wissenschaftliche Paper analysiert und strukturiert aufbereitet, was meine Fähigkeiten in der Datenanalyse und Dokumentation unterstreicht.

Die Aufgaben dieser Position begeistern mich sehr, insbesondere die Möglichkeit, Build- und Release-Pipelines zu erstellen, Automatisierungsskripte zu entwickeln und Dashboards für Telemetrie-Daten zu erstellen. Die Arbeit im Product Experience Team bietet eine einzigartige Gelegenheit, an der Schnittstelle zwischen Technologie und Produkterfahrung zu arbeiten und einen Beitrag zur Verbesserung von SAP-Produkten zu leisten.

Meine Deutschkenntnisse sind sehr gut (C1), und ich verfüge über fließende Englischkenntnisse (C2). Ich bin bereit, regelmäßig (1-2 Mal pro Woche) vor Ort am SAP Walldorf Campus zu arbeiten und bringe hohe Motivation, Selbstständigkeit und Teamgeist mit.

Sehr gerne würde ich in einem persönlichen Gespräch mehr über die Details der Position erfahren und meine Motivation sowie meine Qualifikationen näher darstellen.

\closing{Mit freundlichen Grüßen,}

\end{letter}

\end{document}
