\documentclass[11pt,a4paper]{letter}
\usepackage[utf8]{inputenc}
\usepackage[T1]{fontenc}
\usepackage[ngerman]{babel}
\usepackage{geometry}
\usepackage{hyperref}

% Page margins
\geometry{
  left=2.5cm,
  right=2.5cm,
  top=2cm,
  bottom=2cm
}

% Sender information
\address{Erfan Taghvaei \\
Kaiserslautern, Deutschland \\
+49 157 35338285 \\
etaghvaei0098@gmail.com}

% Date
\date{\today}

\begin{document}

% Recipient
\begin{letter}{%
Academic Work Germany GmbH \\
z. Hd. Adda Hansen und Andre Jelken \\
Düsseldorf \\
Deutschland
}

\opening{Sehr geehrte Frau Hansen, sehr geehrter Herr Jelken,}

mit großem Interesse bewerbe ich mich über Academic Work auf die Position als Softwareentwickler Java. Die Möglichkeit, an geschäftskritischer Individualsoftware für führende Unternehmen zu arbeiten und den gesamten Software-Lifecycle von der Planung bis zum Betrieb zu begleiten, entspricht genau meinen beruflichen Zielen.

Ich verfüge über zwei Jahre Berufserfahrung als Softwareentwickler bei Neocosmo in Deutschland, wo ich in agilen Teams an verschiedenen Kundenprojekten für Unternehmen wie Festo, KVNO, Riniger und die Hochschule Kempten gearbeitet habe. In dieser Zeit habe ich umfassendes Verständnis des gesamten Softwareentwicklungsprozesses entwickelt -- von der Anforderungsanalyse über die Architektur und Implementierung bis zur Dokumentation und dem Betrieb. Besonders prägend war ein Projekt, bei dem ich über zwei Monate eigenverantwortlich ein kritisches Ticket bearbeitet und in direkter Zusammenarbeit mit dem Kunden erfolgreich umgesetzt habe. Meine technischen Kompetenzen umfassen fundierte JavaScript-Kenntnisse durch zwei Jahre intensive Entwicklung mit React Native, umfangreiche Backend-Erfahrung mit Python/Flask sowie praktische Fullstack-Entwicklung, Erfahrung mit modernen Frontend-Technologien (React) durch mehrere eigenständige Projekte wie RecallCards, Arbeit mit Datenbanken (MongoDB, Redis) und Cloud-Technologien, praktische DevOps-Erfahrung mit GitLab CI/CD, Docker-Containerisierung und automatisierten Deployments, Code Reviews und Teststrategien (TDD mit Detox, Storybook) sowie tägliche Arbeit mit Git-Versionskontrolle in agilen Entwicklerteams.

Aktuell studiere ich im Master of Computer Science an der RPTU Kaiserslautern mit Schwerpunkt Software Engineering. Meine akademische Ausbildung umfasst relevante Kurse wie Foundations of Software Engineering, Object-Oriented Programming, Compiler Design und Data Structures and Algorithms. Ich verfüge über verhandlungssichere Deutschkenntnisse (C1) und fließende Englischkenntnisse (C2). Aufgrund meiner soliden Programmiererfahrung mit objektorientierten Sprachen, meinem umfassenden Verständnis von Backend-Frameworks und meiner schnellen Lernfähigkeit bin ich überzeugt, dass ich mir Java und Spring Boot in kürzester Zeit aneignen kann -- wie ich bereits mit verschiedenen Technologien (React Native, Python/Flask, MongoDB) unter Beweis gestellt habe. Meine kürzlich erworbene MongoDB Associate Developer Zertifizierung unterstreicht meine kontinuierliche Weiterbildung. Die Möglichkeit, in vielfältigen Projekten mit wechselnden Technologien zu arbeiten und mich kontinuierlich weiterzuentwickeln, reizt mich sehr. Sehr gerne würde ich Sie in einem persönlichen Gespräch von meinen Fähigkeiten überzeugen.

\closing{Mit freundlichen Grüßen,}

\end{letter}

\end{document}
