\documentclass[11pt,a4paper]{letter}
\usepackage[utf8]{inputenc}
\usepackage[T1]{fontenc}
\usepackage[ngerman]{babel}
\usepackage{geometry}
\usepackage{hyperref}

% Page margins
\geometry{
  left=2.5cm,
  right=2.5cm,
  top=2cm,
  bottom=2cm
}

% Sender information
\address{Erfan Taghvaei \\
Kaiserslautern, Deutschland \\
+49 157 35338285 \\
etaghvaei0098@gmail.com}

% Date
\date{\today}

\begin{document}

% Recipient
\begin{letter}{%
Porsche AG \\
Weissach \\
Deutschland
}

\opening{Sehr geehrtes Porsche-Team,}

mit großem Interesse bewerbe ich mich auf die Masterarbeit zum Thema "Früherkennung von Leichtbaupotentialen durch KI-gestützte Datenanalyse in der Gesamtfahrzeugentwicklung". Die Möglichkeit, innovative KI-Methoden mit der faszinierenden Welt der Fahrzeugentwicklung zu verbinden und dabei einen Beitrag zur Zukunft des Leichtbaus bei Porsche zu leisten, motiviert mich außerordentlich.

Ich studiere aktuell im Master of Computer Science an der RPTU Kaiserslautern mit Schwerpunkt Software Engineering und Data Visualization. Meine akademische Ausbildung umfasst relevante Kurse wie Quantum Computing in AI, Data Visualization, Foundations of Software Engineering sowie fundamentale Kenntnisse in Data Structures and Algorithms. Diese Kombination aus KI-bezogenen Themen und Softwareentwicklung bildet eine solide Basis für die anspruchsvollen Aufgaben dieser Masterarbeit. Während meines Studiums habe ich mich intensiv mit Datenanalyse, Machine Learning Konzepten und der praktischen Anwendung von KI-Methoden auseinandergesetzt. Ich bringe zwei Jahre Berufserfahrung als Softwareentwickler bei Neocosmo in Deutschland mit, wo ich an verschiedenen Kundenprojekten gearbeitet und wichtige Erfahrungen in strukturierter, professioneller Softwareentwicklung gesammelt habe. Meine IT-Kenntnisse umfassen fundierte Programmierkenntnisse in Python, das ich für Backend-Entwicklung mit Flask und Datenverarbeitung eingesetzt habe, praktische Erfahrung mit Datenmodellierung und Datenbankstrukturen durch Arbeit mit MongoDB, Redis und SQL-basierten Systemen, sicherer Umgang mit MS Excel und Office-Anwendungen sowie Erfahrung mit strukturierter Datenverarbeitung und -analyse in meinem AAS Sanity Projekt, wo ich ein System zur automatischen Analyse und Validierung von JSON-Daten mit 18 Regeln entwickelt habe.

Ich bringe ein ausgeprägtes analytisches Denkvermögen mit, das sich in meiner Fähigkeit zeigt, komplexe technische Probleme zu strukturieren und systematisch zu lösen, wie bei meinem Universitätsprojekt mit Multithreading-Optimierung, wo ich ineffiziente Algorithmen analysiert und durch optimierte Ansätze ersetzt habe. Meine selbstständige und strukturierte Arbeitsweise habe ich während meiner zweijährigen Berufserfahrung unter Beweis gestellt, insbesondere bei einem Projekt, bei dem ich über zwei Monate eigenverantwortlich ein kritisches Ticket bearbeitete. Ich verfüge über starke Kommunikationsfähigkeiten und Teamfähigkeit durch direkte Zusammenarbeit mit Kunden, Kollegen und Stakeholdern in agilen Teams. Ich bringe Neugierde und Offenheit für interdisziplinäre Zusammenarbeit mit und bin sehr interessiert daran, mein Wissen aus der Informatik mit den ingenieurwissenschaftlichen Aspekten des Fahrzeugbaus zu verbinden. Meine Deutschkenntnisse sind sehr gut (C1) und ich verfüge über fließende Englischkenntnisse (C2), was mir eine sichere Kommunikation in beiden Sprachen ermöglicht. Obwohl mein Studium primär Informatik und Data Science fokussiert, bin ich hochmotiviert, mich in die technischen Produktstrukturen und Entwicklungslogik der Fahrzeugtechnik einzuarbeiten und sehe darin eine spannende Gelegenheit, mein Wissen zu erweitern.

Die Aufgaben dieser Masterarbeit begeistern mich sehr, insbesondere die Möglichkeit, durch Literaturrecherche, Interviews mit Fachbereichen und die Entwicklung eines KI-gestützten Prototyps zur Datenverknüpfung einen echten Mehrwert für Porsche zu schaffen. Die Kombination aus technischer Datenanalyse, KI-Entwicklung und strategischer Bewertung entspricht genau meinen Interessen und Fähigkeiten. Bei Porsche meine Masterarbeit zu schreiben, wäre für mich eine einzigartige Gelegenheit, bei einem der innovativsten Automobilhersteller der Welt zu lernen und beizutragen. Sehr gerne würde ich in einem persönlichen Gespräch mehr über die Details der Arbeit erfahren und meine Motivation sowie meine Qualifikationen näher darstellen.

\closing{Mit freundlichen Grüßen,}

\end{letter}

\end{document}
