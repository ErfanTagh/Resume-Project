\documentclass[11pt,a4paper]{letter}
\usepackage[utf8]{inputenc}
\usepackage[T1]{fontenc}
\usepackage[ngerman]{babel}
\usepackage{geometry}

% Page geometry
\geometry{
    left=2.5cm,
    right=2.5cm,
    top=2cm,
    bottom=2cm
}

% Sender information
\address{Erfan Taghvaei \\
Kaiserslautern, Germany \\
+49 157 35338285 \\
etaghvaei0098@gmail.com}

% Date
\date{\today}

\begin{document}

\begin{letter}{Sleak GmbH \\
München, Germany}

\subject{Bewerbung als Product Development Engineer - Working Student (Werkstudent)}

\opening{Sehr geehrte Damen und Herren,}

ich bewerbe mich auf die Position als Product Development Engineer - Working Student bei Sleak AI. Ich habe eure Stellenausschreibung gesehen und finde die Idee, Sales-Teams mit conversational AI beim Training zu unterstützen, wirklich spannend.

\textbf{Warum Sleak:}

Was mich an Sleak interessiert, ist dass ihr ein echtes Problem löst. Sales-Reps müssen ständig üben, aber realistische Kundengespräche zu simulieren ist schwierig. Eure Lösung macht das möglich -- und das finde ich cool. Als Startup könnt ihr schnell entscheiden und umsetzen, ohne durch viele Hierarchien zu müssen. Genau das suche ich: Verantwortung übernehmen, direkt mit dem Team arbeiten und Features entwickeln, die wirklich genutzt werden.

\vspace{0.5cm}

\textbf{Was mich am Produkt reizt:}

Die technische Herausforderung finde ich interessant: Eine KI zu bauen, die sich an verschiedene Unternehmen anpassen lässt und realistische Gespräche simuliert. Das ist nicht trivial. Gleichzeitig sehe ich, dass euer Produkt wirklich funktioniert -- Vertriebler können damit üben und werden besser. Das ist kein theoretisches Tool, sondern etwas, das messbare Ergebnisse liefert. Als Entwickler mag ich genau solche Projekte: technisch anspruchsvoll, aber mit klarem Nutzen.

\vspace{0.5cm}

\textbf{Was ich mitbringe:}

Ich studiere Computer Science im Master an der RPTU Kaiserslautern und arbeite seit zwei Jahren als React-Native-Werkstudent bei Neocosmo. Dort entwickle ich hauptsächlich mit TypeScript, React und Python. In meiner Freizeit experimentiere ich mit LLMs, LangChain und Hugging Face -- also genau den Technologien, die ihr für conversational AI braucht.

Bei Neocosmo habe ich gelernt, eigenständig zu arbeiten und direkt mit Kunden zu kommunizieren. Ich entwickle Features, die wirklich genutzt werden, und optimiere Performance, wenn etwas langsam ist. Mir ist wichtig, dass Nutzer das Produkt verstehen und gerne nutzen.

\vspace{0.5cm}

\textbf{Warum ich passe:}

Technisch decke ich ab, was ihr braucht: TypeScript, React (Next.js basiert ja darauf), REST APIs und Full-Stack-Entwicklung. Mit LangChain und LLMs kenne ich mich aus, kann also schnell bei conversational AI-Features mitarbeiten. Cursor nutze ich täglich für die Entwicklung.

Wichtiger finde ich aber: Ich will nicht nur Code schreiben, sondern verstehen, was Nutzer wirklich brauchen. Wenn etwas nicht intuitiv ist, frage ich nach und schlage Verbesserungen vor. In kleinen Teams habe ich gelernt, dass direkte Kommunikation und schnelle Umsetzung wichtig sind -- genau das passt zu einem Startup.

\vspace{0.5cm}

\textbf{Warum ich zu euch will:}

Als Student passt mir die Flexibilität eines Working Student-Verhältnisses perfekt. Ich würde gerne direkt mit euch an der Roadmap arbeiten und Features entwickeln, die Sales-Teams wirklich helfen. Falls es gut läuft, kann ich mir vorstellen, langfristig dabei zu bleiben.

Ich würde mich freuen, wenn wir uns mal unterhalten können.

\closing{Mit freundlichen Grüßen}

\encl{Lebenslauf \\ Zeugnisse \\ Immatrikulationsbescheinigung}

\end{letter}

\end{document}

