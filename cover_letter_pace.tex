\documentclass[11pt,a4paper]{letter}
\usepackage[utf8]{inputenc}
\usepackage[T1]{fontenc}
\usepackage[ngerman]{babel}
\usepackage{geometry}
\usepackage{hyperref}

% Page margins
\geometry{
  left=2.5cm,
  right=2.5cm,
  top=2cm,
  bottom=2cm
}

% Sender information
\address{Erfan Taghvaei \\
Kaiserslautern, Deutschland \\
+49 157 35338285 \\
etaghvaei0098@gmail.com}

% Date
\date{\today}

\begin{document}

% Recipient
\begin{letter}{%
PACE Telematics GmbH \\
Karlsruhe \\
Deutschland
}

\opening{Sehr geehrtes PACE-Team,}

bereit, Gas zu geben? Auf jeden Fall! Mit großer Begeisterung bewerbe ich mich auf die Position als Backend Developer bei PACE Telematics. Die Entwicklung skalierbarer Node.js-Backends für innovative Connected Car Services wie connectedfueling.com ist genau die Art von technischer Herausforderung, die mich antreibt.

\textbf{Warum PACE Telematics:}

Als führender Technologieanbieter im Bereich Connected Car Services verbindet PACE genau die Themen, die mich als Entwickler faszinieren: moderne Backend-Technologien, skalierbare Cloud-Plattformen und innovative Lösungen für reale Anwendungsfälle. Die Arbeit an Plattformen, die täglich von Tausenden Nutzern verwendet werden, motiviert mich besonders.

\textbf{Meine Erfahrung:}

Ich bringe zwei Jahre Berufserfahrung als Softwareentwickler in Deutschland bei Neocosmo mit, wo ich als React Native Entwickler an verschiedenen Kundenprojekten für Unternehmen wie Festo, KVNO, Riniger und die Hochschule Kempten gearbeitet habe. In dieser Zeit habe ich umfangreiche Erfahrung in der Entwicklung von robusten, skalierbaren Anwendungen gesammelt und gelernt, wie wichtig performante und sichere Software-Lösungen sind.

\textbf{Technische Qualifikationen für Ihre Anforderungen:}

\begin{itemize}
  \item \textbf{Backend-Entwicklung}: Mehrere Jahre Erfahrung mit Backend-Technologien -- umfangreiche Projekte mit Flask (Python) und REST-API-Entwicklung. Solide JavaScript-Grundlage durch React Native Entwicklung, bereit für TypeScript/Node.js
  \item \textbf{API-Design}: Praktische Erfahrung im Design und der Implementierung von RESTful APIs für die Kommunikation zwischen Frontend und Backend in meinen Fullstack-Projekten
  \item \textbf{Datenbank-Integration}: Arbeit mit verschiedenen Datenbanken (Redis, MongoDB) und externe Service-Integration in produktiven Anwendungen
  \item \textbf{TypeScript/JavaScript}: Zwei Jahre intensive JavaScript-Entwicklung, TypeScript-Kenntnisse aus verschiedenen Projekten
  \item \textbf{Test-Driven Development}: Erfahrung mit automatisierten Tests (Detox, Storybook) und Unit Testing
  \item \textbf{Code Reviews}: Aktive Teilnahme an Code Reviews zur Sicherstellung hoher Code-Qualität
  \item \textbf{Continuous Delivery}: GitLab CI/CD-Pipelines, automatisierte Deployments, Docker-Containerisierung
  \item \textbf{Git}: Tägliche Arbeit mit Versionskontrollsystemen, Branch-Management, Merge-Strategien
\end{itemize}

\textbf{Praktische Projekterfahrung:}

\begin{itemize}
  \item \textbf{RecallCards Flashcard-App}: Vollständig entwickelte Anwendung mit React-Frontend und Flask-Backend, Auth0-Integration, Redis für Session-Management, Docker-containerisiert mit NGINX und GitHub Actions CI/CD
  \item \textbf{AAS Sanity Projekt}: Backend mit Flask, Multithreading-Optimierung für Performance-Verbesserung, REST-API-Entwicklung
\end{itemize}

\textbf{Was ich mitbringe:}

\begin{itemize}
  \item \textbf{Fokus auf Qualität}: Ich entwickle sichere, stabile und performante Anwendungen mit Fokus auf Best Practices
  \item \textbf{Teamfähigkeit}: Zwei Jahre Erfahrung in der Zusammenarbeit mit Entwicklern, Kunden und Stakeholdern
  \item \textbf{Eigenständiges Arbeiten}: Eigenverantwortliche Projektbearbeitung und zielgerichtete Umsetzung
  \item \textbf{Kommunikationsstärke}: Erfahrung in direkter Kundenkommunikation und Anforderungsklärung
  \item \textbf{Sprachkenntnisse}: Sehr gute Deutschkenntnisse (C1) und fließende Englischkenntnisse (C2)
  \item \textbf{Humor}: Eine gute Portion Humor gehört definitiv dazu -- Arbeit soll auch Spaß machen!
\end{itemize}

Aktuell studiere ich im Master of Computer Science an der RPTU Kaiserslautern mit Schwerpunkten in Software Engineering, Data Visualization und Network Security. Diese Kombination aus praktischer Berufserfahrung und fortgeschrittener akademischer Ausbildung ermöglicht es mir, sowohl technisch fundierte als auch konzeptionell durchdachte Backend-Lösungen zu entwickeln.

Die von Ihnen beschriebenen Benefits -- insbesondere das hybride Arbeitsmodell, die vielfältigen Projekte und die Socializing-Events -- sprechen mich sehr an. Ich suche ein Umfeld, in dem ich meine Backend-Entwicklungsfähigkeiten einbringen, an innovativen Automotive-Technologien arbeiten und gemeinsam mit einem motivierten Team etwas bewegen kann.

\textbf{Verfügbarkeit und Gehaltsvorstellung:}

\begin{itemize}
  \item \textbf{Möglicher Eintrittstermin}: Ab sofort bzw. nach Absprache
  \item \textbf{Gehaltsvorstellung}: Nach Vereinbarung, gerne im persönlichen Gespräch
\end{itemize}

Bereit, den Motor zu starten? Ich bin es auf jeden Fall! Sehr gerne würde ich Sie in einem persönlichen Gespräch von meinen Fähigkeiten überzeugen und mehr über die spannenden Projekte bei PACE Telematics erfahren.

\closing{Mit freundlichen Grüßen,}

\end{letter}

\end{document}
