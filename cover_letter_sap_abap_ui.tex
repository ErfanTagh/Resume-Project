\documentclass[11pt,a4paper]{letter}
\usepackage[utf8]{inputenc}
\usepackage[T1]{fontenc}
\usepackage[ngerman]{babel}
\usepackage{geometry}
\usepackage{hyperref}

% Page margins
\geometry{
  left=2.5cm,
  right=2.5cm,
  top=2cm,
  bottom=2cm
}

\begin{document}

% Sender information
\address{Erfan Taghvaei \\
Kaiserslautern, Deutschland \\
+49 157 35338285 \\
etaghvaei0098@gmail.com}

% Date
\date{\today}

% Recipient
\begin{letter}{%
SAP SE \\
Walldorf \\
Deutschland
}

\opening{Sehr geehrtes SAP ABAP Development Tools Team,}

mit großem Interesse bewerbe ich mich auf die Position als Working Student (f/m/d) für UI/UX Design und Frontend-Entwicklung bei SAP. Die Möglichkeit, an der Modernisierung von Benutzeroberflächen für ABAP-Entwicklungstools mitzuwirken und dabei Design und Software-Entwicklung zu verbinden, motiviert mich außerordentlich.

Ich studiere aktuell im Master of Computer Science an der RPTU Kaiserslautern mit Schwerpunkt Software Engineering und Data Visualization. Meine akademische Ausbildung umfasst relevante Kurse wie Data Visualization, Foundations of Software Engineering sowie Social Web Mining (Note: 1.7). Besonders hervorheben möchte ich mein DFKI Smart Factory Projekt, bei dem ich eine vollständige Full-Stack-Anwendung mit modernen Webtechnologien und benutzerfreundlichem Design entwickelt habe.

Meine praktische Erfahrung als React Native Entwickler bei Neocosmo (Feb 2022 - Feb 2024) hat mir wertvolle Einblicke in die Frontend-Entwicklung und UI/UX-Design gegeben. Ich habe eigenverantwortlich komplexe Tickets bearbeitet, dabei direkt mit Kunden kommuniziert und erfolgreich JavaScript-Code zu TypeScript konvertiert. Diese Erfahrung hat meine Fähigkeiten in der Benutzeroberflächen-Entwicklung, Datenmodellierung und kontinuierlichen Kommunikation gestärkt.

Meine technischen Kompetenzen umfassen fundierte Programmierkenntnisse in TypeScript und JavaScript, die ich sowohl in meinem DFKI-Projekt als auch in meiner Forschungsarbeit "COVID-19 personal protective equipment detection using real-time deep learning methods" (arXiv:2103.14878) eingesetzt habe. In dieser Arbeit habe ich mit YOLO und SSD MobileNet-Algorithmen gearbeitet und eine Genauigkeit von 90.6\% erreicht. Diese Erfahrung mit Datenverarbeitung und Visualisierung ist besonders relevant für Ihre UI/UX-Entwicklung.

Ich verfüge über umfassende Erfahrung mit Frontend-Programmierung und modernen Webtechnologien, die ich sowohl in meinem DFKI-Projekt als auch in meiner Berufserfahrung bei Neocosmo eingesetzt habe. In meinem Circular Economy Seminar habe ich über 30 wissenschaftliche Paper analysiert und strukturiert aufbereitet, was meine Fähigkeiten in der Datenvisualisierung und benutzerfreundlichen Darstellung unterstreicht.

Die Aufgaben dieser Position begeistern mich sehr, insbesondere die Möglichkeit, an der Modernisierung von Benutzeroberflächen mitzuwirken, visuelles Design zu unterstützen und Frontend-Implementierungen mit TypeScript zu entwickeln. Die Arbeit im ABAP Development Tools Team bietet eine einzigartige Gelegenheit, an der Schnittstelle zwischen Design und Software-Entwicklung zu arbeiten und einen Beitrag zur Verbesserung der Entwicklererfahrung zu leisten.

Meine Deutschkenntnisse sind sehr gut (C1), und ich verfüge über fließende Englischkenntnisse (C2). Ich bin bereit, regelmäßig (2-3 Mal pro Woche) vor Ort am SAP Walldorf Campus zu arbeiten und bringe hohe Motivation, Selbstständigkeit, Kreativität und Teamgeist mit.

Sehr gerne würde ich in einem persönlichen Gespräch mehr über die Details der Position erfahren und meine Motivation sowie meine Qualifikationen näher darstellen.

\closing{Mit freundlichen Grüßen,}

\end{letter}

\end{document}
