\documentclass[11pt,a4paper]{letter}
\usepackage[utf8]{inputenc}
\usepackage[T1]{fontenc}
\usepackage[ngerman]{babel}
\usepackage{geometry}
\usepackage{hyperref}

% Page margins
\geometry{
  left=2.5cm,
  right=2.5cm,
  top=2cm,
  bottom=2cm
}

\begin{document}

% Sender information
\address{Erfan Taghvaei \\
Kaiserslautern, Deutschland \\
+49 157 35338285 \\
etaghvaei0098@gmail.com}

% Date
\date{\today}

% Recipient
\begin{letter}{%
DATEV eG \\
Nürnberg \\
Deutschland
}

\opening{Sehr geehrtes DATEV Team,}

mit großem Interesse bewerbe ich mich auf die Position als Werkstudent Softwareentwicklung (m/w/d) bei DATEV. Die Möglichkeit, an der Entwicklung der Accounting Online Plattform von morgen mitzuwirken und dabei Cloud Services und intelligente Datenverarbeitung zu realisieren, motiviert mich außerordentlich.

Ich studiere aktuell im Master of Computer Science an der RPTU Kaiserslautern mit Schwerpunkt Software Engineering und Data Visualization. Meine akademische Ausbildung umfasst relevante Kurse wie Data Visualization, Foundations of Software Engineering sowie Social Web Mining (Note: 1.7). Besonders hervorheben möchte ich mein DFKI Smart Factory Projekt, bei dem ich eine vollständige Full-Stack-Anwendung mit modernen Webtechnologien und Cloud-Integration entwickelt habe.

Meine praktische Erfahrung als React Native Entwickler bei Neocosmo (Feb 2022 - Feb 2024) hat mir wertvolle Einblicke in die agile Softwareentwicklung und Cross-Functional Teamarbeit gegeben. Ich habe eigenverantwortlich komplexe Tickets über zwei Monate bearbeitet, dabei direkt mit Kunden kommuniziert und erfolgreich JavaScript-Code zu TypeScript konvertiert. Diese Erfahrung hat meine Fähigkeiten in der API-Entwicklung, Datenmodellierung und kontinuierlichen Kommunikation gestärkt.

Meine technischen Kompetenzen umfassen fundierte Programmierkenntnisse in TypeScript, JavaScript und Java, die ich sowohl in meinem DFKI-Projekt als auch in meiner Berufserfahrung bei Neocosmo eingesetzt habe. Ich verfüge über umfassende Erfahrung mit Git, IntelliJ und modernen Entwicklungstools sowie Verständnis für Clean-Code-Prinzipien und automatisierte Tests.

Mein AAS Sanity Projekt zeigt meine Fähigkeiten in der Datenanalyse und Systementwicklung mit automatischer Validierung von JSON-Daten. Diese Erfahrung mit Datenverarbeitung und Monitoring ist besonders relevant für Ihre Datentransfer-Tools und Reporting-Systeme.

Die Aufgaben dieser Position begeistern mich sehr, insbesondere die Implementierung von Cloud Services, die Entwicklung responsiver Frontend-Oberflächen und die Weiterentwicklung von Monitoring-Tools. Die Arbeit im POS-Services Team bietet eine einzigartige Gelegenheit, an der Schnittstelle zwischen Cloud-Technologie und intelligenter Datenverarbeitung zu arbeiten und einen Beitrag zur Digitalisierung der Geschäftsprozesse zu leisten.

Meine Deutschkenntnisse sind sehr gut (C1), und ich verfüge über fließende Englischkenntnisse (C2). Ich bin bereit, remote zu arbeiten und bringe hohe Motivation, Selbstständigkeit, Teamgeist und Eigeninitiative mit.

Sehr gerne würde ich in einem persönlichen Gespräch mehr über die Details der Position erfahren und meine Motivation sowie meine Qualifikationen näher darstellen.

\closing{Mit freundlichen Grüßen,}

\end{letter}

\end{document}
