\documentclass[11pt,a4paper]{letter}
\usepackage[utf8]{inputenc}
\usepackage[T1]{fontenc}
\usepackage[ngerman]{babel}
\usepackage{geometry}
\usepackage{hyperref}

% Page margins
\geometry{
  left=2.5cm,
  right=2.5cm,
  top=2cm,
  bottom=2cm
}

% Sender information
\address{Erfan Taghvaei \\
Kaiserslautern, Deutschland \\
+49 157 35338285 \\
etaghvaei0098@gmail.com}

% Date
\date{\today}

\begin{document}

% Recipient
\begin{letter}{%
growney GmbH \\
Berlin \\
Deutschland
}

\opening{Sehr geehrtes growney-Team,}

mit großer Begeisterung bewerbe ich mich auf die Position als Working Student Software Development bei growney. Die Möglichkeit, an einer echten digitalen Transformation mitzuwirken und von einem PHP-Monolithen zu einer modernen, cloud-nativen Serverless-Architektur zu migrieren, ist genau die Art von herausfordernder Lernerfahrung, die ich suche. Darüber hinaus bin ich sehr motiviert, nach Berlin umzuziehen, da ich Berlin als eine deutlich lebendigere und dynamischere Stadt als Kaiserslautern empfinde und die Stadt bessere Möglichkeiten für mein berufliches Wachstum und meine persönliche Entwicklung bietet.

Ich studiere aktuell im Master of Computer Science an der RPTU Kaiserslautern mit Schwerpunkt Software Engineering. Ich bringe zwei Jahre Berufserfahrung als Softwareentwickler bei Neocosmo in Deutschland mit, wo ich als React Native Entwickler an verschiedenen Kundenprojekten für Unternehmen wie Festo, KVNO, Riniger und die Hochschule Kempten gearbeitet habe. In dieser Zeit habe ich praktische Erfahrung mit dem gesamten Softwareentwicklungszyklus in agilen Teams gesammelt, von der Anforderungsanalyse bis zum produktiven Deployment. Meine technischen Kompetenzen umfassen umfangreiche praktische Erfahrung mit JavaScript und TypeScript durch zwei Jahre React Native Entwicklung, solide React-Kenntnisse für Webanwendungen, die ich in meinem RecallCards-Projekt unter Beweis gestellt habe, wo ich eine vollständige Fullstack-Anwendung mit React-Frontend entwickelt habe, Backend-Entwicklungserfahrung mit Flask und REST-APIs, die mir eine starke Grundlage für die Arbeit mit Node.js bietet, praktische Erfahrung mit MySQL und relationalen Datenbanken sowie NoSQL-Datenbanken wie MongoDB und Redis, und praktische DevOps-Erfahrung inklusive Docker-Containerisierung für konsistente Deployments, GitLab CI/CD-Pipelines für automatisierte Tests und Deployments sowie Arbeit mit Cloud-Technologien und Deployment-Prozessen.

Ich bin besonders begeistert von growneys Tech-Transformation, weil ich die Herausforderungen und Chancen der Modernisierung von Legacy-Systemen verstehe. Meine Erfahrung mit Docker-Containerisierung, Microservice-Denken durch meine Fullstack-Projekte und CI/CD-Automatisierung passt sehr gut zu Ihrer Migration zu einer cloud-nativen Architektur. Ich bin motiviert zu lernen, detailorientiert und begeistert davon, zu echten Produktionssystemen beizutragen. Meine MongoDB Associate Developer Zertifizierung von 2024 zeigt mein Engagement für kontinuierliches Lernen und das Bleiben auf dem neuesten Stand moderner Technologien. Ich bin ein Teamplayer, der es genießt, komplexe technische Herausforderungen zu lösen, und habe Erfahrung in der Zusammenarbeit mit Entwicklern und Stakeholdern in agilen Umgebungen. Ich spreche und schreibe fließend Englisch und verfüge über sehr gute Deutschkenntnisse. Am wichtigsten ist, dass ich bereit und begeistert bin, nach Berlin umzuziehen und Teil der lebendigen Tech-Szene dort zu werden. In Berlin zu arbeiten wäre ein bedeutender Schritt nach vorne für meine Karriere, und ich bin voll und ganz engagiert, diesen Umzug zu machen, um Teil von growneys ambitioniertem Team zu werden.

Ich bin überzeugt, dass meine praktische Softwareentwicklungserfahrung, meine Motivation, moderne Cloud-Technologien zu lernen, und meine echte Begeisterung für Berlin und growneys Mission mich zu einer wertvollen Ergänzung für Ihr Team machen. Ich würde mich sehr freuen, in einem persönlichen Gespräch zu besprechen, wie ich zur Transformation Ihrer Plattform beitragen kann.

\closing{Mit freundlichen Grüßen,}

\end{letter}

\end{document}
