\documentclass[11pt,a4paper]{letter}
\usepackage[utf8]{inputenc}
\usepackage[T1]{fontenc}
\usepackage[ngerman]{babel}
\usepackage{geometry}
\usepackage{hyperref}

% Page margins
\geometry{
  left=2.5cm,
  right=2.5cm,
  top=2cm,
  bottom=2cm
}

% Sender information
\address{Erfan Taghvaei \\
Kaiserslautern, Deutschland \\
+49 157 35338285 \\
etaghvaei0098@gmail.com}

% Date
\date{\today}

\begin{document}

% Recipient
\begin{letter}{%
A11 \\
Berlin \\
Deutschland
}

\opening{Sehr geehrtes A11-Team,}

mit großer Begeisterung bewerbe ich mich auf die Position als Frontend Developer bei A11. Die Möglichkeit, die Frontend-Entwicklung für ein innovatives Unternehmen aufzubauen und die Verlagswelt ins digitale Zeitalter zu führen, begeistert mich sehr. Ich bin bereit, nach Berlin umzuziehen und Teil Ihres dynamischen Ecosystem zu werden.

Ich verfüge über zwei Jahre Berufserfahrung als Frontend-Entwickler bei Neocosmo in Deutschland, wo ich an verschiedenen Kundenprojekten für Unternehmen wie Festo, KVNO, Riniger und die Hochschule Kempten gearbeitet habe. In dieser Zeit habe ich umfangreiche Erfahrung in der Entwicklung von modernen, responsiven Benutzeroberflächen mit React Native gesammelt und war eigenverantwortlich für Frontend-Komponenten verantwortlich. Besonders prägend war ein Projekt, bei dem ich über zwei Monate eigenverantwortlich ein kritisches Ticket bearbeitet und kreative Lösungen entwickelt habe. Zusätzlich habe ich mehrere Fullstack-Projekte eigenständig entwickelt, darunter RecallCards mit React-Frontend, wo ich die gesamte User Experience von Grund auf gestaltet habe. Meine technischen Kompetenzen umfassen zwei Jahre praktische Erfahrung mit React, das konzeptionell sehr ähnlich zu Angular ist, solide TypeScript-Kenntnisse aus verschiedenen Projekten, Backend-Verständnis durch Arbeit mit Python/Flask und REST-APIs, was mir ermöglicht, Symfony-Code zu lesen und zu verstehen, praktische Erfahrung mit Unit- und E2E-Tests (Detox, Storybook), GitLab CI/CD und kontinuierliche Integration sowie eigenständige Entwicklung kreativer UI/UX-Lösungen ohne ausschließliche Abhängigkeit von Libraries.

Aktuell studiere ich im Master of Computer Science an der RPTU Kaiserslautern mit Schwerpunkt Software Engineering. Ich verfüge über fließende Deutschkenntnisse in Wort und Schrift sowie sehr gute Englischkenntnisse. Meine Frontend-first-Mentalität zeigt sich darin, dass ich großen Wert auf User Experience lege und kreative, durchdachte Lösungen entwickle. Aufgrund meiner soliden React-Erfahrung und meiner schnellen Lernfähigkeit bin ich überzeugt, dass ich mir Angular in kürzester Zeit aneignen kann -- die Konzepte von Component-based Development, State Management und Routing sind mir bereits vertraut. Ebenso kann ich mich schnell in Symfony einarbeiten, da ich bereits umfangreiche Backend-Erfahrung mit Flask und REST-APIs mitbringe. Ich bin ein Teamplayer, arbeite eigenständig und strukturiert und bringe eigene Ideen ein, um Produkte kontinuierlich zu verbessern. Die Möglichkeit, Hauptverantwortung für die gesamte Frontend-Entwicklung zu übernehmen und direkten Einfluss auf das technische Design zu haben, reizt mich sehr. Sehr gerne würde ich Sie in einem persönlichen Gespräch von meinen Fähigkeiten überzeugen.

\closing{Mit freundlichen Grüßen,}

\end{letter}

\end{document}
