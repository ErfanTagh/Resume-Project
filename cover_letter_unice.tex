\documentclass[11pt,a4paper]{letter}
\usepackage[utf8]{inputenc}
\usepackage[T1]{fontenc}
\usepackage[ngerman]{babel}
\usepackage{geometry}
\usepackage{hyperref}

% Page margins
\geometry{
  left=2.5cm,
  right=2.5cm,
  top=2cm,
  bottom=2cm
}

% Sender information
\address{Erfan Taghvaei \\
Kaiserslautern, Deutschland \\
+49 157 35338285 \\
etaghvaei0098@gmail.com}

% Date
\date{\today}

\begin{document}

% Recipient
\begin{letter}{%
uNaice \\
Deutschland
}

\opening{Sehr geehrtes uNaice Team,}

mit großem Interesse bewerbe ich mich auf die Position als Junior Backend Developer (Python) bei uNaice. Die Möglichkeit, an innovativen Lösungen für automatisierte Contentproduktion und Datenaufbereitung mitzuwirken und dabei mit Top-Playern wie adidas, TUI und C&A zusammenzuarbeiten, motiviert mich außerordentlich.

Ich studiere aktuell im Master of Computer Science an der RPTU Kaiserslautern mit Schwerpunkt Software Engineering und Data Visualization. Meine akademische Ausbildung umfasst relevante Kurse wie Data Visualization, Foundations of Software Engineering sowie Social Web Mining (Note: 1.7). Besonders hervorheben möchte ich mein DFKI Smart Factory Projekt, bei dem ich eine vollständige Full-Stack-Anwendung mit Python Backend entwickelt habe. Diese Erfahrung hat mir gezeigt, wie wichtig performante APIs und strukturierte Datenverarbeitung sind.

Meine praktische Erfahrung als React Native Entwickler bei Neocosmo (Feb 2022 - Feb 2024) hat mir wertvolle Einblicke in die Backend-Entwicklung gegeben. Ich habe eigenverantwortlich ein großes Ticket über zwei Monate bearbeitet, dabei direkt mit Kunden kommuniziert und erfolgreich JavaScript-Code zu TypeScript konvertiert. Diese Erfahrung hat meine Fähigkeiten in der API-Entwicklung, Datenmodellierung und kontinuierlichen Kommunikation gestärkt.

Meine technischen Kompetenzen umfassen fundierte Programmierkenntnisse in Python, die ich sowohl in meinem DFKI-Projekt als auch in meiner Forschungsarbeit "COVID-19 personal protective equipment detection using real-time deep learning methods" (arXiv:2103.14878) eingesetzt habe. In dieser Arbeit habe ich mit YOLO und SSD MobileNet-Algorithmen gearbeitet und eine Genauigkeit von 90.6\% erreicht. Diese Erfahrung mit Datenverarbeitung und Machine Learning ist besonders relevant für Ihre DataContentNaicer-Lösung.

Ich verfüge über Erfahrung mit verschiedenen Datenformaten (JSON, CSV, XML) und habe in meinem Circular Economy Seminar über 30 wissenschaftliche Paper analysiert und strukturiert aufbereitet. Meine Kenntnisse in Git, Linux und modernen Entwicklungstools, kombiniert mit meiner Fähigkeit, komplexe Datenpipelines zu verstehen und zu optimieren, machen mich zu einem idealen Kandidaten für diese Position.

Die Aufgaben dieser Position begeistern mich sehr, insbesondere die Möglichkeit, Backend-Services mit Python zu entwickeln, Datenquellen zu integrieren und Datenpipelines aufzubauen. Die Arbeit an DataContentNaicer bietet eine einzigartige Gelegenheit, an der Schnittstelle zwischen Datenverarbeitung und Content-Automatisierung zu arbeiten und einen Beitrag zur Digitalisierung großer Unternehmen zu leisten.

Meine Deutschkenntnisse sind sehr gut (C1), und ich verfüge über fließende Englischkenntnisse (C2). Ich bin bereit, remote zu arbeiten und bringe hohe Motivation, Selbstständigkeit und Teamgeist mit. Die flexible Arbeitsweise und die Möglichkeit, in einem innovationsgetriebenen Produktteam mitzuarbeiten, entspricht genau meinen Vorstellungen von einer zukunftsorientierten Karriere.

Sehr gerne würde ich in einem persönlichen Gespräch mehr über die Details der Position erfahren und meine Motivation sowie meine Qualifikationen näher darstellen.

\closing{Mit freundlichen Grüßen,}

\end{letter}

\end{document}
