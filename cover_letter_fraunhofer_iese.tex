\documentclass[11pt,a4paper]{letter}
\usepackage[utf8]{inputenc}
\usepackage[T1]{fontenc}
\usepackage[ngerman]{babel}
\usepackage{geometry}
\usepackage{hyperref}

% Page margins
\geometry{
  left=2.5cm,
  right=2.5cm,
  top=2cm,
  bottom=2cm
}

% Sender information
\address{Erfan Taghvaei \\
Kaiserslautern, Deutschland \\
+49 157 35338285 \\
etaghvaei0098@gmail.com}

% Date
\date{\today}

\begin{document}

% Recipient
\begin{letter}{%
Marc Lorenz \\
Fraunhofer-Institut für Experimentelles Software Engineering IESE \\
Kaiserslautern \\
Deutschland
}

\opening{Sehr geehrter Herr Lorenz,}

mit großem Interesse bewerbe ich mich auf die Masterarbeit zum Thema "Extending the Predictive Autonomy Lab" am Fraunhofer IESE. Die Möglichkeit, an der Entwicklung innovativer Sicherheitstechnologien für automatisierte Fahrzeuge mitzuwirken und dabei praktische Erfahrungen in einem der führenden Forschungsumgebungen Deutschlands zu sammeln, motiviert mich außerordentlich.

Ich studiere aktuell im Master of Computer Science an der RPTU Kaiserslautern mit Schwerpunkt Software Engineering und Data Visualization. Meine akademische Ausbildung umfasst relevante Kurse wie Quantum Computing in AI, Data Visualization, Foundations of Software Engineering sowie Social Web Mining (Note: 1.7). Besonders hervorheben möchte ich mein DFKI Smart Factory Projekt, bei dem ich eine vollständige Full-Stack-Anwendung mit Dockerisierung und umfassender Dokumentation entwickelt habe. Diese Erfahrung hat mir gezeigt, wie wichtig systematische Integration und Validierung komplexer Systeme ist.

Meine praktische Erfahrung als React Native Entwickler bei Neocosmo (Feb 2022 - Feb 2024) hat mir wertvolle Einblicke in die Softwareentwicklung gegeben. Ich habe eigenverantwortlich ein großes Ticket über zwei Monate bearbeitet, dabei direkt mit Kunden kommuniziert und erfolgreich JavaScript-Code zu TypeScript konvertiert. Diese Erfahrung hat meine Fähigkeiten in der Problemlösung, Systemintegration und kontinuierlichen Kommunikation gestärkt.

Meine technischen Kompetenzen umfassen fundierte Programmierkenntnisse in Python und C++, die ich sowohl in meinem Robotics Team während des Bachelors als auch in meiner Forschungsarbeit "COVID-19 personal protective equipment detection using real-time deep learning methods" (arXiv:2103.14878) eingesetzt habe. In dieser Arbeit habe ich mit YOLO und SSD MobileNet-Algorithmen gearbeitet und eine Genauigkeit von 90.6\% erreicht. Diese Erfahrung mit Computer Vision und Deep Learning ist besonders relevant für Driver Monitoring Systems.

Meine Deutschkenntnisse sind sehr gut (C1), und ich verfüge über fließende Englischkenntnisse (C2). Ich bin bereit, vor Ort am Fraunhofer IESE in Kaiserslautern zu arbeiten und bringe hohe Motivation, Selbstständigkeit und Zuverlässigkeit mit. Meine Erfahrung mit Git, Linux und modernen Entwicklungstools, kombiniert mit meiner Fähigkeit, komplexe technische Konzepte zu erklären und Feedback zu empfangen, macht mich zu einem idealen Kandidaten für diese Masterarbeit.

Die Aufgaben dieser Masterarbeit begeistern mich sehr, insbesondere die Möglichkeit, eine systematische Umfrage zu Open-Source-Systemen durchzuführen, eine geeignete Lösung auszuwählen und eine technische Integration zu implementieren. Die Arbeit am Predictive Autonomy Lab bietet eine einzigartige Gelegenheit, an der Schnittstelle zwischen Forschung und praktischer Anwendung zu arbeiten und einen Beitrag zur Weiterentwicklung sicherheitsbasierter Technologien im automatisierten Fahren zu leisten.

Sehr gerne würde ich in einem persönlichen Gespräch mehr über die Details der Arbeit erfahren und meine Motivation sowie meine Qualifikationen näher darstellen.

\closing{Mit freundlichen Grüßen,}

\end{letter}

\end{document}
