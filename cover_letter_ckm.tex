\documentclass[11pt,a4paper]{letter}
\usepackage[utf8]{inputenc}
\usepackage[T1]{fontenc}
\usepackage[ngerman]{babel}
\usepackage{geometry}
\usepackage{hyperref}

% Page margins
\geometry{
  left=2.5cm,
  right=2.5cm,
  top=2cm,
  bottom=2cm
}

% Sender information
\address{Erfan Taghvaei \\
Kaiserslautern, Deutschland \\
+49 157 35338285 \\
etaghvaei0098@gmail.com}

% Date
\date{\today}

\begin{document}

% Recipient
\begin{letter}{%
CKM Group \\
Berlin Coding Company \\
Berlin \\
Deutschland
}

\opening{Sehr geehrtes CKM-Team,}

mit großer Begeisterung bewerbe ich mich auf die Position als Junior Frontend Developer bei der Berlin Coding Company. Die Möglichkeit, an sinnvollen, gesellschaftsrelevanten Produkten im Gesundheitsbereich zu arbeiten und durch digitale Lösungen die Gesundheitsversorgung von morgen mitzugestalten, motiviert mich sehr.

Ich verfüge über zwei Jahre Berufserfahrung als Frontend-Entwickler bei Neocosmo in Deutschland, wo ich an verschiedenen Kundenprojekten für Unternehmen wie Festo, KVNO, Riniger und die Hochschule Kempten gearbeitet habe. Meine Qualifikationen übertreffen die Junior-Anforderungen deutlich. Ich habe umfangreiche praktische Erfahrung mit klassischen Webtechnologien durch zwei Jahre professionelle Entwicklung mit HTML, CSS, JavaScript und modernen Frontend-Frameworks wie React Native sowie eigenständige Webprojekte mit React. Ich besitze fundierte Kenntnisse in SASS und Bootstrap durch verschiedene Projekte, in denen ich responsive, strukturierte Oberflächen entwickelt habe. Ich habe Erfahrung mit serverseitigem Rendering und Backend-Integration durch meine Fullstack-Projekte mit Flask, wo ich Template-Rendering und die Verbindung von Funktionalität und Layout umgesetzt habe. Meine Arbeitsweise ist sauber und strukturiert, ich habe ein ausgeprägtes Gespür für UI und UX und lege großen Wert auf Details, wie mein RecallCards-Projekt zeigt, wo ich die gesamte User Experience gestaltet habe. Ich habe praktische Erfahrung mit Git-Versionierung durch tägliche Arbeit bei Neocosmo, NPM als Package-Manager, Build-Tools und CI/CD-Prozessen durch GitLab-Pipelines und GitHub Actions sowie der Einarbeitung in bestehende Codebasen durch zwei Jahre Arbeit in großen Projekten mit mehreren Entwicklern.

Aktuell studiere ich im Master of Computer Science an der RPTU Kaiserslautern mit Schwerpunkt Software Engineering. Ich verfüge über fließende Deutschkenntnisse (C1) und sehr gute Englischkenntnisse (C2). Ich bin motiviert, mich in bestehende Projekte und Codebasen einzuarbeiten, und bringe die Bereitschaft mit, kontinuierlich zu lernen und mich weiterzuentwickeln. Die Möglichkeit, an Produkten zu arbeiten, die echten gesellschaftlichen Impact haben und zur Verbesserung der Gesundheitsversorgung beitragen, begeistert mich besonders. Ihre Werte von Innovation, Kreativität und progressivem Denken passen perfekt zu meiner Arbeitsweise. Sehr gerne würde ich Sie in einem persönlichen Gespräch von meinen Fähigkeiten überzeugen.

\closing{Mit freundlichen Grüßen,}

\end{letter}

\end{document}
