\documentclass[11pt,a4paper]{letter}
\usepackage[utf8]{inputenc}
\usepackage[T1]{fontenc}
\usepackage[ngerman]{babel}
\usepackage{geometry}
\usepackage{hyperref}

% Page margins
\geometry{
  left=2.5cm,
  right=2.5cm,
  top=2cm,
  bottom=2cm
}

% Sender information
\address{Erfan Taghvaei \\
Kaiserslautern, Deutschland \\
+49 157 35338285 \\
etaghvaei0098@gmail.com}

% Date
\date{\today}

\begin{document}

% Recipient
\begin{letter}{%
Sensaru GmbH \\
Karlsruhe \\
Deutschland
}

\opening{Sehr geehrtes Sensaru-Team,}

mit großer Begeisterung bewerbe ich mich auf die Position als Frontend Entwickler Junior bei Sensaru. Die Möglichkeit, an intelligenter Heizungssteuerung zu arbeiten und durch innovative Technologie aktiv zur Energiewende beizutragen, begeistert mich sehr. Visionäre Ideen in handfeste Lösungen zu verwandeln und technologische Grenzen zu verschieben, ist genau das, was mich als Entwickler antreibt.

Ich verfüge über zwei Jahre Berufserfahrung als Frontend-Entwickler bei Neocosmo in Deutschland, wo ich an verschiedenen Kundenprojekten für Unternehmen wie Festo, KVNO, Riniger und die Hochschule Kempten gearbeitet habe. Meine Qualifikationen übertreffen die Anforderungen deutlich, da ich nicht nur Grundkenntnisse, sondern fundierte praktische Erfahrung mitbringe. Ich habe umfangreiche Erfahrung mit JavaScript und TypeScript durch zwei Jahre professionelle React Native Entwicklung sowie eigenständige React-Webprojekte. Ich verstehe den Unterschied zwischen UI und UX und habe aktiv an der Gestaltung von Anforderungen und Designentscheidungen mitgewirkt, insbesondere bei meinem RecallCards-Projekt, wo ich die gesamte User Experience von Grund auf gestaltet habe. Ich habe ein gutes Verständnis von Software Design Patterns und habe diese in verschiedenen Projekten angewendet, um Standardprobleme effizient zu lösen. Meine Erfahrung mit Test Driven Development umfasst praktische Arbeit mit Unit- und E2E-Tests durch Detox und Storybook, und ich verstehe die Vor- und Nachteile dieses Ansatzes. Ich kenne Software-Engineering-Prozesse wie das V-Modell und agile Methoden wie SCRUM und Kanban aus meiner zweijährigen Arbeit in agilen Teams. Ich arbeite täglich mit modernen IDEs wie VS Code und bin vertraut mit verschiedenen Entwicklungsumgebungen. Ich habe praktische Erfahrung mit Continuous Integration und Deployment durch GitLab CI/CD-Pipelines, Docker-Containerisierung und GitHub Actions, und ich verstehe die gesamte Toolchain sowie deren Vorteile für moderne Softwareentwicklung.

Aktuell studiere ich im Master of Computer Science an der RPTU Kaiserslautern mit Schwerpunkt Software Engineering, was meine praktische Erfahrung mit solider theoretischer Basis ergänzt. Ich bin bereit, meine Fähigkeiten kontinuierlich zu verbessern und Neues zu lernen, wie meine kürzlich erworbene MongoDB Associate Developer Zertifizierung zeigt. Ich kenne und lebe die Lebensregel, dass es immer jemanden gibt, der besser ist, und sehe das als Motivation, ständig von anderen zu lernen und mich weiterzuentwickeln. Meine Leidenschaft für die Softwareentwicklung zeigt sich darin, dass ich auch in meiner Freizeit programmiere und eigene Projekte entwickle. Ich arbeite mit Kreativität, Teamgeist und dem Wunsch, nicht nur Code zu schreiben, sondern echten Impact zu erzielen. Die Möglichkeit, bei Sensaru an smarten Lösungen für die Energiewende zu arbeiten und dabei in Karlsruhe zu bleiben, ist für mich eine ideale Kombination. Sehr gerne würde ich Sie in einem persönlichen Gespräch von meinen Fähigkeiten überzeugen.

\closing{Mit freundlichen Grüßen,}

\end{letter}

\end{document}
