\documentclass[11pt,a4paper]{letter}
\usepackage[utf8]{inputenc}
\usepackage[T1]{fontenc}
\usepackage[ngerman]{babel}
\usepackage{geometry}
\usepackage{hyperref}

% Page margins
\geometry{
  left=2.5cm,
  right=2.5cm,
  top=2cm,
  bottom=2cm
}

% Sender information
\address{Erfan Taghvaei \\
Kaiserslautern, Deutschland \\
+49 157 35338285 \\
etaghvaei0098@gmail.com}

% Date
\date{\today}

\begin{document}

% Recipient
\begin{letter}{%
Fraunhofer-Institut für Experimentelles Software Engineering IESE \\
Kaiserslautern \\
Deutschland
}

\opening{Sehr geehrtes Fraunhofer IESE Team,}

mit großem Interesse bewerbe ich mich auf die Position als studentische Hilfskraft im technischen Bereich (Facility-Management) am Fraunhofer IESE. Die Möglichkeit, während meines Studiums praktische Erfahrungen in einem der führenden Forschungsumgebungen Deutschlands zu sammeln und aktiv an innovativen Lösungen für nachhaltige digitale Ökosysteme mitzuwirken, motiviert mich außerordentlich.

Ich studiere aktuell im Master of Computer Science an der RPTU Kaiserslautern mit Schwerpunkt Software Engineering und Data Visualization. Meine akademische Ausbildung umfasst relevante Kurse wie Data Visualization, Foundations of Software Engineering sowie Social Web Mining (Note: 1.7). Besonders hervorheben möchte ich mein Circular Economy Seminar, bei dem ich über 30 wissenschaftliche Paper analysiert und bewertet habe. Diese Erfahrung hat mir ein tiefes Verständnis für Nachhaltigkeit und Umweltschutz vermittelt und mein Interesse an innovativen Lösungen für eine nachhaltigere Zukunft geweckt.

Meine praktische Erfahrung als React Native Entwickler bei Neocosmo (Feb 2022 - Feb 2024) hat mir wertvolle Einblicke in strukturierte Arbeitsweise und Projektmanagement gegeben. Ich habe eigenverantwortlich ein großes Ticket über zwei Monate bearbeitet, dabei direkt mit Kunden kommuniziert und erfolgreich komplexe technische Probleme gelöst. Diese Erfahrung hat meine Fähigkeiten in der Datenanalyse, Problemlösung und kontinuierlichen Kommunikation gestärkt.

Meine technischen Kompetenzen umfassen fundierte Kenntnisse in MS Office-Programmen, die ich sowohl in meiner Berufserfahrung als auch in meinen Studienprojekten regelmäßig einsetze. In meinem DFKI Smart Factory Projekt habe ich umfassende Dokumentation mit README-Dateien und Sequenzdiagrammen erstellt, was meine Fähigkeiten in der strukturierten Datenaufbereitung und -darstellung unterstreicht.

Meine Deutschkenntnisse sind sehr gut (C1), und ich verfüge über fließende Englischkenntnisse (C2). Ich bin bereit, vor Ort am Fraunhofer IESE in Kaiserslautern zu arbeiten und bringe hohe Motivation, Selbstständigkeit und Serviceorientierung mit. Meine Erfahrung mit Git, Linux und modernen Entwicklungstools, kombiniert mit meiner Fähigkeit, komplexe Aufgabenstellungen zu analysieren und Lösungsansätze zu erarbeiten, macht mich zu einem idealen Kandidaten für diese Position.

Die Aufgaben dieser Position begeistern mich sehr, insbesondere die Möglichkeit, bei der Einführung und Aufrechterhaltung des Energiemanagementsystems (ISO 50001) mitzuwirken und Daten zu erheben und aufzuarbeiten. Die Arbeit am Fraunhofer IESE bietet eine einzigartige Gelegenheit, praktische Erfahrungen in der angewandten Forschung zu sammeln und einen Beitrag zur Gestaltung verlässlicher digitaler Ökosysteme zu leisten.

Sehr gerne würde ich in einem persönlichen Gespräch mehr über die Details der Position erfahren und meine Motivation sowie meine Qualifikationen näher darstellen.

\closing{Mit freundlichen Grüßen,}

\end{letter}

\end{document}
