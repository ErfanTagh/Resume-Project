\documentclass[11pt,a4paper]{letter}
\usepackage[utf8]{inputenc}
\usepackage[T1]{fontenc}
\usepackage[ngerman]{babel}
\usepackage{geometry}
\usepackage{hyperref}

% Page margins
\geometry{
  left=2.5cm,
  right=2.5cm,
  top=2cm,
  bottom=2cm
}

% Sender information
\address{Erfan Taghvaei \\
Kaiserslautern, Deutschland \\
+49 157 35338285 \\
etaghvaei0098@gmail.com}

% Date
\date{\today}

\begin{document}

% Recipient
\begin{letter}{%
Nanoscribe GmbH \& Co. KG \\
Karlsruhe \\
Deutschland
}

\opening{Sehr geehrtes Nanoscribe-Team,}

mit großem Interesse habe ich Ihre Stellenausschreibung für die Position als Software Engineer/Developer Python gelesen. Die Möglichkeit, an hochpräziser 3D-Mikrofabrikation zu arbeiten und innovative Software-Lösungen für modernste Hardware zu entwickeln, begeistert mich sehr.

\textbf{Warum Nanoscribe:}

Als Technologie- und Weltmarktführer in der additiven Fertigung bietet Nanoscribe genau das Umfeld, in dem ich meine Leidenschaft für strukturierten Code und technische Innovation einbringen möchte. Die Arbeit an einer komplexen Software-Architektur, die von Touch UI bis zu numerischen Algorithmen und Hardware-Steuerung reicht, ist eine Herausforderung, die mich begeistert.

\textbf{Meine technische Qualifikation:}

Ich bringe zwei Jahre Berufserfahrung als Softwareentwickler in Deutschland bei Neocosmo mit, wo ich an verschiedenen Kundenprojekten für Unternehmen wie Festo, KVNO, Riniger und die Hochschule Kempten gearbeitet habe. Besonders stolz bin ich auf ein Projekt, bei dem ich über zwei Monate eigenverantwortlich ein kritisches Ticket bearbeitet habe und in direkter Zusammenarbeit mit dem Kunden alle Fehler erfolgreich beheben konnte.

\textbf{Relevante technische Fähigkeiten:}

\begin{itemize}
  \item \textbf{Python-Entwicklung}: Umfangreiche Erfahrung mit Python und Flask -- mehrere Projekte eigenständig entwickelt, darunter eine Webanwendung mit Multithreading-Optimierung und ein Flashcard-System mit Backend-Logik
  \item \textbf{Arbeit in großen Codebasen}: Zwei Jahre Erfahrung bei Neocosmo in komplexen React Native Projekten mit mehreren Entwicklern
  \item \textbf{Client/Server-Architektur}: Fullstack-Projekte mit Frontend (React) und Backend (Flask) sowie REST-API-Kommunikation
  \item \textbf{Multi-Threading}: Praktische Erfahrung mit Multithreading-Optimierung in einem Universitätsprojekt zur Performance-Verbesserung
  \item \textbf{Linux \& CI/CD}: Tägliche Arbeit mit Linux-Systemen, GitLab CI/CD-Pipelines, Docker-Containerisierung
  \item \textbf{Build-Tools \& DevOps}: Erfahrung mit modernen Build-Prozessen, automatisierter Testing (Detox, Storybook) und CI-Integration
  \item \textbf{Git}: Tägliche Arbeit mit Versionskontrollsystemen und Code Reviews
\end{itemize}

\textbf{Analytische Fähigkeiten:}

Während meines Studiums und meiner Projekte habe ich ein tiefes Verständnis für Algorithmen entwickelt. In meinem AAS Sanity Projekt habe ich beispielsweise einen Algorithmus mit Multithreading implementiert und statt des langsamen JSON-Deepcopy einen optimierten Ansatz mit Code-Snapshots und Referenzen entwickelt. Diese Erfahrung zeigt meine Fähigkeit, Algorithmen zu analysieren, zu verstehen und zu optimieren.

Aktuell studiere ich im Master of Computer Science an der RPTU Kaiserslautern mit Schwerpunkten in Software Engineering, Data Visualization und Network Security. Meine akademische Ausbildung umfasst relevante Kurse wie Compiler Design, Data Structures and Algorithms und Operating Systems -- genau die theoretische Basis, die für die Entwicklung komplexer Software-Systeme wichtig ist.

\textbf{Persönliches Profil:}

\begin{itemize}
  \item \textbf{Sauberer Code}: Ich lege großen Wert auf wartbaren, testbaren Code und beteilige mich aktiv an Code Reviews
  \item \textbf{Teamarbeit}: Zwei Jahre Erfahrung in der engen Zusammenarbeit mit Entwicklern, Kunden und Stakeholdern
  \item \textbf{Lernbereitschaft}: Schnelle Einarbeitung in neue Technologien -- kürzlich MongoDB Associate Developer Zertifizierung erworben
  \item \textbf{Motivation}: Ich programmiere auch privat und bin begeistert von technischen Herausforderungen
  \item \textbf{Sprachkenntnisse}: Sehr gute Deutschkenntnisse (C1) und fließende Englischkenntnisse (C2)
  \item \textbf{Arbeitsgenehmigung}: Als Student an der RPTU verfüge ich über eine gültige Aufenthaltserlaubnis für die Arbeit in Deutschland
\end{itemize}

Die von Ihnen beschriebene Arbeitsumgebung, in der Vielfalt gefeiert wird und alle Talente zum gemeinsamen Erfolg beitragen, spricht mich sehr an. Ich suche genau ein solches Umfeld, in dem ich meine Ideen einbringen und mit Ingenieuren und Domain-Experten zusammenarbeiten kann.

Die Möglichkeit, an der Schnittstelle zwischen Software und hochpräziser Hardware zu arbeiten, Algorithmen über verschiedene Software-Schichten zu optimieren und zur Entwicklung zukunftsweisender 3D-Drucktechnologie beizutragen, ist für mich eine ideale Herausforderung.

Ich bin überzeugt, dass meine Python-Kenntnisse, meine Erfahrung mit komplexen Software-Systemen und meine analytischen Fähigkeiten eine wertvolle Bereicherung für Ihr Team sein können.

Sehr gerne würde ich Sie in einem persönlichen Gespräch von meinen Fähigkeiten überzeugen und mehr über die spannenden Projekte bei Nanoscribe erfahren.

\closing{Mit freundlichen Grüßen}

\end{letter}

\end{document}
